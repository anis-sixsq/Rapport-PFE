\setcounter{mtc}{5} %indique le num�ro r�el du chapitre, pour la mini table des mati�res
\chapter{Pr�sentation et cadre du projet}
\minitoc  %insert la minitoc

\graphicspath{{Chapitre0/figures/}}
%==============================================================================
\pagestyle{fancy}
\fancyhf{}
\fancyhead[R]{\bfseries\rightmark}
\fancyfoot[R]{\thepage}
\renewcommand{\headrulewidth}{0.5pt}
\renewcommand{\footrulewidth}{0pt}
\renewcommand{\chaptermark}[1]{\markboth{\MakeUppercase{\chaptername~\thechapter. #1 }}{}}
\renewcommand{\sectionmark}[1]{\markright{\thechapter.\thesection~ #1}}

\begin{spacing}{1.2}
%==============================================================================

\section*{Introduction}
Une �tude th�orique \cite{YOUSFI2015} peut contenir l'une et/ou l'autre de ces deux parties :

\section{Pr�sentation de l'entreprise 'SixSq'} 


\section{Cloud Computing et Big Data}
On va pr�senter les domaines :  Cloud et Big Data

\section{M�thodologie de travail} 
On va pr�senter la m�thodologie Scrum + Une comparison avec les autres m�thodologies + Outlis utilis�s


\section*{Conclusion}
La conclusion est en g�n�ral sans num�rotation, et n'appara�t pas dans la table des mati�res.


%==============================================================================
\end{spacing}
