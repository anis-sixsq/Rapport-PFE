\setcounter{mtc}{5} %indique le num�ro r�el du chapitre, pour la mini table des mati�res
\chapter{Pr�sentation et cadre du projet}
\minitoc  %insert la minitoc

\graphicspath{{Chapitre0/figures/}}
%==============================================================================
\pagestyle{fancy}
\fancyhf{}
\fancyhead[R]{\bfseries\rightmark}
\fancyfoot[R]{\thepage}
\renewcommand{\headrulewidth}{0.5pt}
\renewcommand{\footrulewidth}{0pt}
\renewcommand{\chaptermark}[1]{\markboth{\MakeUppercase{\chaptername~\thechapter. #1 }}{}}
\renewcommand{\sectionmark}[1]{\markright{\thechapter.\thesection~ #1}}

\begin{spacing}{1.2}
%==============================================================================

\section*{Introduction}
Une �tude th�orique \cite{YOUSFI2015} peut contenir l'une et/ou l'autre de ces deux parties :

TEST �����  TEST 

\section{Pr�sentation de l'entreprise 'SixSq'} 
SixSq est un leader europ\'{e}en dans le Cloud Computing qui fournit des solutions aux entreprises nationales et internationales de toutes tailles. L'entreprise se sp\'{e}cialise dans l'automatisation des processus, apportant des avantages financiers \`{a} ces clients via ces produits uniques: SlipStream\textregistered{} et NuvlaBox\textregistered{}. Son \'{e}quipe, qui se compose d'ing\'{e}nieurs de logiciels hautement qualifi\'{e}s, d\'{e}veloppeurs et administrateurs syst\`{e}me de 10 pays diff\'{e}rents, est bas\'{e} \`{a} Gen\`{e}ve, en Suisse.

\\Partenariats
\\SixSq collabore et participe \`{a} plusieurs programmes de partenariats. Et voici un r\'{e}sum\'{e} de certaines des relations  les plus importantes qu'elle~a~\'{e}tabli~: 
\\\textbf{EXOSCALE}: SixSq est le partenaire technologique d'Exoscale, le principal fournisseur de services de cloud suisse. Avec le connecteur SlipStream, les clients peuvent d\'{e}ployer leurs applications vers le cloud Exoscale en un seul clic.
\\\textbf{AMAZON}: SixSq est un fournisseur de solutions Amazon, avec un service de SlipStream d\'{e}di\'{e} et configur\'{e} pour d\'{e}ployer les applications sur le service EC2.  
\\\textbf{IBM}: SixSq est membre du programme \guillemotleft{}~IBM Partner World~\guillemotright{}. Elle a  certifi\'{e} SlipStream sur des solutions mat\'{e}rielles et  logicielles IBM. 
\\\textbf{Helix Nebula}: SixSq est un membre fondateur de la collaboration Helix Nebula, qui est un partenariat novateur entre les chercheurs scientifiques et les entreprises en Europe. 
\\\textbf{RHEA}~: SixSq forme un partenariat strat\'{e}gique avec RHEA, le premier fournisseur europ\'{e}en de services d'ing\'{e}nierie de syst\`{e}mes et de solutions logicielles pour les domaines~: l'a\'{e}rospatiale, la d\'{e}fense et  la s\'{e}curit\'{e} informatique.

\\Recherche et d\'{e}veloppement
\\SixSq est \`{a} la fronti\`{e}re entre le d\'{e}veloppement innovant et l'exploitation commerciale. SixSq est n\'{e} d'id\'{e}es cr\'{e}\'{e}es au CERN, l'organisation europ\'{e}enne pour la recherche nucl\'{e}aire. SixSq continue \`{a} participer \`{a} des projets de recherche et d\'{e}veloppement, \`{a} la fois aux niveaux national et international. 
\\\textbf{StratusLab} : est un logiciel de cloud IaaS (Infrastructure as a Service) issu d'un projet europ\'{e}en (FP7) et d\'{e}velopp\'{e} par une communaut\'{e} open-source.
\\\textbf{SCISSOR}  : con\c{c}oit un nouveau cadre de surveillance de la s\'{e}curit\'{e} SCADA de g\'{e}n\'{e}ration pour permettre, syst\`{e}mes connect\'{e}s, encore s\'{e}curis\'{e}s industriels contr\^{o}le. 
\\\textbf{PaaSword} : La S\'{e}curit\'{e} et la protection des donn\'{e}es sont  des obstacles importants \`{a} une large utilisation de plates-formes de cloud computing. PaaSword vise \`{a} r\'{e}duire ces obstacles en fournissant, le stockage s\'{e}curis\'{e} sur les services de cloud computing.
\\\textbf{CYCLONE} : d\'{e}veloppe une solution pour la gestion de la demande compl\`{e}te dynamique multi-cloud \`{a} partir de composants, de la production de qualit\'{e} existants. La solution comprend la gestion automatis\'{e}e de l'application, la mise en r\'{e}seau de pointe, une s\'{e}curit\'{e} de bout-en-bout, et de la gestion d'identit\'{e} f\'{e}d\'{e}r\'{e}e.  
\\\textbf{CELAR} : L'objectif du projet de CELAR est de d\'{e}velopper des m\'{e}thodes et des outils open-source pour l'application et le contr\^{o}le multi-grains, le provisionnement des ressources \'{e}lastique pour des applications de Cloud de mani\`{e}re automatis\'{e}e.

\\Clients et R\'{e}f\'{e}rences
\\Ces clients sont de grandes et petites int\'{e}grateurs de syst\`{e}mes, des soci\'{e}t\'{e}s de haute technologie, les institutions et les organisations internationales. Elle appr\'{e}cie des relations transparentes, fond\'{e}es sur la confiance et l'honn\^{e}tet\'{e}. Et Parmi ses clients on trouve:
\\\textbf{Atos} est une soci\'{e}t\'{e} internationale de services de technologie de l'information avec chiffre d'affaires 8,5 milliards d'euros en 2011 et 74 000 employeurs dans 42 pays.  Il est le partenaire informatique mondial des Jeux olympiques et paralympiques et est cot\'{e}e sur le march\'{e} Eurolist de Paris.
\\\textbf{Citrix Systems} ~est une~entreprise~multinationale~am\'{e}ricaine fond\'{e}e en 1989~qui propose des produits de collaboration, de virtualisation et de mise en r\'{e}seau pour faciliter le travail mobile et l'adoption des services cloud. Citrix compte plus de 330~000  entreprises clientes dans le monde entier.
\\\textbf{L'Union europ\'{e}enne de radio-t\'{e}l\'{e}vision}~est une organisation internationale cr\'{e}\'{e}e en 1950, la plus importante association professionnelle de radiodiffuseurs nationaux dans le monde avec 75 membres actifs dans 56 pays d'Europe, d'Afrique du Nord~et du~Proche-Orient.
\\\textbf{L'Agence spatiale europ\'{e}enne}, \'{e}galement d\'{e}sign\'{e}e sous on acronyme anglais ESA (European Space Agency), est une agence spatiale intergouvernementale coordonnant les projets spatiaux men\'{e}s en commun par une vingtaine de pays europ\'{e}ens. L'agence spatiale, qui par son budget 4 433 millions d'euros en 2015 est la troisi\`{e}me agence spatiale dans le monde apr\`{e}s la NASA et l'agence spatiale f\'{e}d\'{e}rale russe.
\\\textbf{Le~Centre europ\'{e}en des op\'{e}rations spatiales}~(en~anglais: ~European Space Operations Centre~:~ESOC), situ\'{e} \`{a}~Darmstadt~en Allemagne, est charg\'{e} du suivi de toutes les~sondes spatiales~qui sont sous le contr\^{o}le total de l'Agence spatiale europ\'{e}enne~(ESA).
\\\textbf{L'INFN} - l'Institut National de Physique Nucl\'{e}aire - est un institut d\'{e}di\'{e} \`{a} l'\'{e}tude des constituants fondamentaux de la mati\`{e}re, et m\`{e}ne des recherches th\'{e}oriques etexp\'{e}rimentales dans les domaines de subnucl\'{e}aire, nucl\'{e}aire et la physique des astroparticules.
\\\textbf{Interoute} est une~entreprise~de~t\'{e}l\'{e}communications~qui  poss\`{e}de le plus grand r\'{e}seau de nouvelle g\'{e}n\'{e}ration couvrant l'Union europ\'{e}enne, de Londres \`{a} Varsovie, de Stockholm \`{a} la Sicile et au-del\`{a}  dans les \'{e}conomies \'{e}mergentes du continent, y compris la Turquie, avec des stations d'atterrissage de dix sous-marins qui couvrent le pourtour de l'Europe. A l'Ouest, les liens r\'{e}seau \`{a} p\^{o}le majeur de t\'{e}l\'{e}communications en Am\'{e}rique du Nord. A l'Est, le r\'{e}seau relie l'Asie, \`{a} Hong Kong et au Moyen-Orient, via Duba\"{\i}, en Europe. Dans le Sud, l'Afrique, du Cap-town \`{a} Tunis se connecte directement \`{a} l'Europe par Interoute.
\\\textbf{Le Groupe SciSys} est un d\'{e}veloppeur de premier plan de services de TIC, e-Business et des solutions technologiques de pointe qui op\`{e}re dans un large \'{e}ventail de secteurs de march\'{e}, y compris l'espace, les services publics, de la d\'{e}fense, le gouvernement, la communication, les services aux entreprises, les m\'{e}dias et la diffusion et le transport.
\\\textbf{Thales Alenia Space}: leader europ\'{e}en des syst\`{e}mes satellitaires et acteur majeur des infrastructures orbitales, Thales Alenia Space est une joint-venture entre Thales et Finmeccanica et forme avec Telespazio une Alliance spatiale. Thales Alenia Space est une r\'{e}f\'{e}rence mondiale dans les t\'{e}l\'{e}communications, observation radar et optique de la Terre.
\\\textbf{Le CERN}, l'organisation europ\'{e}enne pour la recherche nucl\'{e}aire, est l'un des plus grands et des plus prestigieux laboratoires scientifiques du monde. Il a pour vocation la physique fondamentale, la d\'{e}couverte des constituants et des lois de l'Univers. Comment l'univers a-t-il commenc\'{e}? Les physiciens du CERN cherchent des r\'{e}ponses \`{a} cette question , en utilisant les acc\'{e}l\'{e}rateurs de particules les plus puissants.







\section{Cloud Computing et Big Data}
Le Cloud computing est un mod\`{e}le ~en \'{e}volution et ses d\'{e}finitions, cas d'usages, technologies, avantages et risques seront progressivement affin\'{e}es. L'industrie du Cloud Computing comporte un tr\`{e}s vaste \'{e}cosyst\`{e}me de mod\`{e}les, de fournisseurs et de march\'{e}s sp\'{e}cialis\'{e}s.~
Le Cloud Computing est un mod\`{e}le Informatique qui permet un acc\`{e}s facile et \`{a} la demande par le r\'{e}seau \`{a} un ensemble partag\'{e} de ressources informatiques configurables (serveurs, stockage, applications et services) qui peuvent \^{e}tre rapidement provisionn\'{e}es et lib\'{e}r\'{e}es par ~un minimum d'efforts de gestion ou d'interaction avec le fournisseur du service.
Le mod\`{e}le du Cloud Computing privil\'{e}gie la haute disponibilit\'{e}. Il se compose de~cinq ~caract\'{e}ristiques essentielles, de~trois mod\`{e}les de service~et de~quatre mod\`{e}les de d\'{e}ploiement.

\\
\\Les cinq caract\'{e}ristiques essentielles du Cloud computing
\\L'utilisation de ressources \`{a} distance n'est pas nouveau. Le \textquotedblleft{}time sharing\textquotedblright{} -utilisation partag\'{e}e d'un ordinateur en langage Basic- avait fait son apparition en 1966. On parlait alors de la \guillemotleft{}~prise de calcul~\guillemotright{} \`{a} c\^{o}t\'{e} de la prise de courant.. D\`{e}s le d\'{e}but des ann\'{e}es 70, les activit\'{e}s \textquotedblleft{}service bureau\textquotedblright{} ou \textquotedblleft{}traitement \`{a} fa\c{c}on\textquotedblright{} partageaient des traitements comme les payes ou les facturations sur des infrastructures communes avec souvent une facturation \`{a} l'usage. Plus r\'{e}cemment, sous le nom \textquotedblleft{}Outsourcing\textquotedblright{}, l'h\'{e}bergement et l'exploitation des applications des entreprises \`{a} distance se sont largement d\'{e}velopp\'{e}s. Ces activit\'{e}s n'avaient pas chang\'{e} l'architecture des syst\`{e}mes. Les gains provenaient d'une mise en commun de locaux et de moyens humains et techniques sp\'{e}cialis\'{e}s dans l'exploitation de syst\`{e}mes et d'applications ~existantes.
\\Le mod\`{e}le Cloud Computing se diff\'{e}rencie par les cinq caract\'{e}ristiques essentielles suivantes~:
\\\textbf{Acc\`{e}s aux services par l'utilisateur \`{a} la demande} : La mise en \oe{}uvre des syst\`{e}mes est enti\`{e}rement automatis\'{e}e et c'est l'utilisateur, au moyen d'une console de commande, qui met en place et g\`{e}re la configuration \`{a} distance.
\\\textbf{Acc\`{e}s r\'{e}seau large bande} : Ces centres de traitement sont g\'{e}n\'{e}ralement raccord\'{e}s directement sur le backbone Internet pour b\'{e}n\'{e}ficier d'une excellente connectivit\'{e}. Les grands fournisseurs r\'{e}partissent les centres de traitement sur la plan\`{e}te pour fournir un acc\`{e}s aux syst\`{e}mes en moins de 50 ms de n'importe quel endroit.
\\\textbf{R\'{e}servoir de ressources (non localis\'{e}es)} : La plupart de ces centres comportent des dizaines de milliers de serveurs et de moyens de stockage pour permettre des mont\'{e}es en charge rapides. Il est souvent possible de choisir une zone g\'{e}ographique pour mettre les donn\'{e}es \textquotedblleft{}pr\`{e}s\textquotedblright{} des utilisateurs.
\\\textbf{Redimensionnement rapide (\'{e}lasticit\'{e})} : La mise en ligne d'une nouvelle instance d'un serveur est r\'{e}alis\'{e}e en quelques minutes, l'arr\^{e}t et le red\'{e}marrage en quelques secondes. Toutes ces op\'{e}rations peuvent s'effectuer automatiquement par des scripts. Ces m\'{e}canismes de gestion permettent de b\'{e}n\'{e}ficier pleinement de la facturation \`{a} l'usage en adaptant la puissance de calcul au trafic instantan\'{e}.
\\\textbf{Facturation \`{a} l'usage} : Il n'y a g\'{e}n\'{e}ralement pas de co\^{u}t de mise en service (c'est l'utilisateur qui r\'{e}alise les op\'{e}rations). La facturation est calcul\'{e}e en fonction de la dur\'{e}e et de la quantit\'{e} de ressources utilis\'{e}es. Une unit\'{e} de traitement stopp\'{e}e n'est pas factur\'{e}e.


\\Les trois mod\`{e}les de services
\\Trois mod\`{e}les de services peuvent \^{e}tre offerts sur le Cloud :~Software as a Service~(SaaS),~Platform as a Service~(PaaS), ~Infrastructure as a Service~(IaaS).
Ces trois mod\`{e}les de service doivent \^{e}tre d\'{e}ploy\'{e}s sur des Infrastructures qui poss\`{e}dent les cinq caract\'{e}ristiques essentielles cit\'{e}es plus haut pour \^{e}tre consid\'{e}r\'{e}es comme du Cloud Computing.

\\\textbf{Software as a Service (SaaS)} : Ce mod\`{e}le de service est caract\'{e}ris\'{e}e par l'utilisation d'une application partag\'{e}e qui fonctionne sur une infrastructure Cloud. L'utilisateur acc\`{e}de ~\`{a} l'application par le r\'{e}seau au travers de divers types de terminaux (souvent via un navigateur web). L'administrateur de l'application ne g\`{e}re pas et ne contr\^{o}le pas l'infrastructure sous-jacente (r\'{e}seaux, serveurs, applications, stockage). Il ne contr\^{o}le pas les fonctions de l'application \`{a} l'exception d'un param\'{e}trage de quelques fonctions utilisateurs limit\'{e}es.

\\\textbf{Platform as a Service (PaaS)} : L'utilisateur a la possibilit\'{e} de cr\'{e}er et de d\'{e}ployer sur une infrastructure Cloud PaaS ses propres applications en utilisant les langages et les outils du fournisseur. .L'utilisateur ne g\`{e}re pas ou ne contr\^{o}le pas l'infrastructure Cloud sous-jacente (r\'{e}seaux, serveurs, stockage) mais l'utilisateur contr\^{o}le l'application d\'{e}ploy\'{e}e et sa configuration.


\\\textbf{Infrastructure as a Service (IaaS)} : L'utilisateur loue des moyens de calcul et de stockage, des capacit\'{e}s r\'{e}seau et d'autres ressources indispensables (partage de charge, pare-feu, cache). L'utilisateur a la possibilit\'{e} de d\'{e}ployer n'importe quel type de logiciel incluant les syst\`{e}mes d'exploitation.
L'utilisateur ne g\`{e}re pas ou ne contr\^{o}le pas l'infrastructure Cloud sous-jacente mais il a le contr\^{o}le sur les syst\`{e}mes d'exploitation, le stockage et les applications. Il peut aussi choisir les caract\'{e}ristiques principales des \'{e}quipements r\'{e}seau comme le partage de charge, les pare-feu, etc.

\\
\\\textbf{Les quatre mod\`{e}les de d\'{e}ploiement}
\\Certains distinguent quatre mod\`{e}les de d\'{e}ploiement. Nous les citons ci-apr\`{e}s bien que ces mod\`{e}les n'aient que peu ~d'influence sur les caract\'{e}ristiques techniques des syst\`{e}mes d\'{e}ploy\'{e}es.



\\\textbf{Cloud priv\'{e}}
L'infrastructure Cloud est utilis\'{e}e par une seule organisation. Elle peut \^{e}tre g\'{e}r\'{e}e par l'organisation ou par une tierce partie. L'infrastructure peut \^{e}tre plac\'{e}e dans les locaux de l'organisation ou \`{a} l'ext\'{e}rieur

\\\textbf{Cloud communautaire}
L'infrastructure Cloud est partag\'{e}e par plusieurs organisations pour les besoins d'une communaut\'{e} qui souhaite mettre en commun des moyens (s\'{e}curit\'{e}, conformit\'{e}, etc..). Elle peut \^{e}tre g\'{e}r\'{e}e par les organisations ou par une tierce partie et peut \^{e}tre plac\'{e}e ~dans les locaux ou \`{a} l'ext\'{e}rieur.

\\\textbf{Cloud public}
L'infrastructure cloud est ouverte au public ou \`{a} de grands groupes industriels. Cette infrastructure est poss\'{e}d\'{e}e par une organisation qui vend des services Cloud. C'est le cas le plus courant. C'est celui de la plate-forme Amazon Web Services d\'{e}j\`{a} cit\'{e}e.

\\\textbf{Cloud hybride}
L'infrastructure Cloud est compos\'{e}e d'un ou plusieurs mod\`{e}les ~ci-dessus qui restent des entit\'{e}s s\'{e}par\'{e}es. Ces infrastructures sont li\'{e}es entre elles par la m\^{e}me technologie qui autorise la portabilit\'{e} des applications et des donn\'{e}es. ~C'est une excellente solution pour r\'{e}partir ses moyens en fonction des avantages recherch\'{e}s.


\\\textbf{Les caract\'{e}ristiques communes des diff\'{e}rents mod\`{e}les de Cloud}
\\Le Cloud computing tire parti d'un certain nombre de caract\'{e}ristiques pour fournir des services dans des conditions techniques et \'{e}conomiques tr\`{e}s avantageuses. C'est un peu comme la production d'\'{e}lectricit\'{e}. La plupart des entreprises et des particuliers ont int\'{e}r\^{e}t \`{a} utiliser des fournisseurs dont c'est le m\'{e}tier pour garantir la fiabilit\'{e} et les meilleures conditions \'{e}conomiques. Parmi ces caract\'{e}ristiques communes, on trouve g\'{e}n\'{e}ralement~:

\\\textbf{Des infrastructures gigantesques}
\\Par exemple, le syst\`{e}me de stockage en ligne~Amazon S3~est pass\'{e} de 14 milliards d'objets en janvier 2008 \`{a} 905 milliards d'objets en mars~ 2012 ce qui entra\^{\i}ne des prix de plus en plus bas : de l'ordre de 0.1 euro par giga-octets et par mois.

\\\textbf{Une grande homog\'{e}n\'{e}it\'{e} des moyens}
\\Les syst\`{e}mes regroupent des milliers de composants identiques ce qui simplifie la gestion, la fiabilit\'{e}, ~l'audit et la s\'{e}curit\'{e}.

\\\textbf{virtualisation}
\\La virtualisation est une caract\'{e}ristique indispensable qui pr\'{e}sente de tr\`{e}s nombreux avantages. Le mat\'{e}riel est remplac\'{e} par du logiciel avec tous les avantages du logiciel~: cr\'{e}er une nouvelle machine ou sauvegarder son \'{e}tat consiste \`{a} copier un fichier d'o\`{u} un \'{e}norme gain de temps et d'argent. La machine virtuelle ne tombe pratiquement jamais en panne ce qui accroit s\'{e}rieusement la fiabilit\'{e} des syst\`{e}mes. On peut continuer \`{a} utiliser des machines qui ne sont plus fabriqu\'{e}es. Le pourcentage d'utilisation r\'{e}elle d'un serveur physique d\'{e}passe rarement 15\%. Sur la m\^{e}me puissance de calcul on peut faire fonctionner plusieurs serveurs. Lorsqu'une configuration est utilis\'{e}e pour des d\'{e}veloppements, des op\'{e}rations de recette ou des tests de charge, il est possible de lib\'{e}rer des ressources en archivant la configuration. La remise en ligne du syst\`{e}me lorsque c'est n\'{e}cessaire se fait en quelques minutes.

\\\textbf{Elasticit\'{e} }
\\L'ensemble des caract\'{e}ristiques pr\'{e}c\'{e}dentes (taille, homog\'{e}n\'{e}it\'{e} et virtualisation) permet d'adapter automatiquement la capacit\'{e} de traitement d'une application \`{a} la demande constat\'{e}e. La mise en ligne d'un nouveau serveur peut s'effectuer en moins d'une minute. Il n'est plus n\'{e}cessaire de s'\'{e}quiper pour absorber des pointes de trafic.


\\\textbf{Co\^{u}ts de logiciels tr\`{e}s r\'{e}duits}
\\La plupart des plates-formes publiques utilisent des logiciels open source gratuits. Les co\^{u}ts des logiciels propri\'{e}taires sont souvent factur\'{e}s \`{a} l'usage sans n\'{e}cessiter l'achat de licences. La plupart de ces logiciels sont d\'{e}j\`{a} pr\'{e}install\'{e}s et pr\'{e}configur\'{e}s ce qui fait gagner beaucoup de temps comme expliqu\'{e} dans l'exemple d'utilisation du dernier chapitre.


\\\textbf{Distribution g\'{e}ographique}
\\Les grandes plates-formes publiques disposent de centres r\'{e}partis sur la plan\`{e}te pour r\'{e}duire les risques et placer les donn\'{e}es au plus pr\`{e}s des utilisateurs.


\\\textbf{Orientation Service}
\\Les fonctions fournies aux utilisateurs se pr\'{e}sentent sous la forme de~Web services (REST)~faciles \`{a} utiliser dans un navigateur ou mieux par des scripts automatis\'{e}s. Des groupes de standardisation se sont cr\'{e}\'{e}s pour d\'{e}finir des interfaces communes et simplifier ainsi le passage d'une infrastructure \`{a} une autre.


\\\textbf{Fonctions de s\'{e}curit\'{e} avanc\'{e}es}
\\la s\'{e}curit\'{e} est une pr\'{e}occupation majeure des organisations qui utilisent les services du Cloud. Ces plates-formes disposent g\'{e}n\'{e}ralement de nombreux syst\`{e}mes de protection, hors de port\'{e}e des moyens de la plupart des organisations.














\section{M�thodologie de travail} 
On va pr�senter la m�thodologie Scrum + Une comparison avec les autres m�thodologies + Outlis utilis�s


\section*{Conclusion}
La conclusion est en g�n�ral sans num�rotation, et n'appara�t pas dans la table des mati�res.


%==============================================================================
\end{spacing}
