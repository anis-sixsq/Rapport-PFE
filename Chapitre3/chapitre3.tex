
\setcounter{chapter}{3}
\chapter{R�alisation}
\minitoc %insert la minitoc
\graphicspath{{Chapitre3/figures/}}

%\DoPToC
%==============================================================================
\pagestyle{fancy}
\fancyhf{}
\fancyhead[R]{\bfseries\rightmark}
\fancyfoot[R]{\thepage}
\renewcommand{\headrulewidth}{0.5pt}
\renewcommand{\footrulewidth}{0pt}
\renewcommand{\chaptermark}[1]{\markboth{\MakeUppercase{\chaptername~\thechapter. #1 }}{}}
\renewcommand{\sectionmark}[1]{\markright{\thechapter.\thesection~ #1}}

\begin{spacing}{1.2}

%==============================================================================
\section*{Introduction}
Ce chapitre porte sur la partie pratique

\section{Outils et langages utilis�s}
L'�tude technique peut se trouver dans cette partie, comme elle peut �tre faite en
parall�le avec l'�tude th�orique (comme le sugg�re le mod�le 2TUP).
Dans cette partie, il faut essayer de convaincre le lecteur de vos choix en termes de
technologie. Un �tat de l'art est souhait� ici, avec un comparatif, une synth�se et un choix 
d'outils, m�me tr�s brefs.

\section{Pr�sentation de l'application}
Il est tout � fait normal que tout le monde attende cette partie pour coller � souhait toutes les images
correspondant aux interfaces diverses de l'application si ch�re � votre coeur, mais
abstenez vous! Il FAUT mettre des imprime �crans, mais bien choisis, et surtout, il faut les sc�nariser : Choisissez un sc�nario d'ex�cution, par exemple la cr�ation d'un 
nouveau client, et montrer les diff�rentes interfaces n�cessaires pour le faire, en
expliquant bri�vement le comportement de l'application. Pas trop d'images, ni trop de
commentaires : concis, encore et toujours.

�vitez ici de coller du code : personne n'a envie de voir le contenu de vos classes.
Mais  vous  pouvez ins�rer des snippets (bouts de code) pour montrer certaines
fonctionnalit�s \cite{YOUSFI2015}\cite{Latex}, si vous en avez vraiment besoin. Si vous voulez montrer une partie de votre code, les �tapes d'installation ou de configuration, vous pourrez les mettre dans l'annexe.


 

\section*{Conclusion}
Voil�.

%==============================================================================
\end{spacing}
