\chapter{Conclusion G�n�rale et Perspectives}
%==============================================================================
\pagestyle{fancy}
\fancyhf{}
\fancyhead[R]{\bfseries\rightmark}
\fancyfoot[R]{\thepage}
\renewcommand{\headrulewidth}{0.5pt}
\renewcommand{\footrulewidth}{0pt}
\renewcommand{\chaptermark}[1]{\markboth{\MakeUppercase{\chaptername~\thechapter. #1 }}{}}
\renewcommand{\sectionmark}[1]{\markright{\thechapter.\thesection~ #1}}

\begin{spacing}{1.2}
%==============================================================================
Notre mission dans ce stage consiste � r�aliser et mettre en place un module Big Data au sein de la plateforme SlipStream. Cette mission a �t� propos�e par l'entreprise SixSq dont le but est de fournir � ses clients une solution Big Data pr�te � d�ployer dans la plupart des infrastructures Cloud Computing existent sur le march� et aussi de monter la performance de SlipStream qui supporte les d�ploiements multi-cloud et multi-machine d'une mani�re efficace et optimale.\\\\

Apr�s la r�alisation de ce projet, les clients de SixSq, qui sont principalement des scientifiques et des chercheurs dans les grandes institutions et organisations internationales tel que le centre europ�en pour la recherche nucl�aire et l'agence spatiale europ�enne, peuvent utiliser SlipStream pour d�ployer notre solution dans un environnement de Cloud Computing. Cette solution est constitu�e d'un cluster Hadoop param�trable, des consoles d'administration et d'autres outils pour connecter et utiliser Hadoop.\\\\

Bien �videmment que notre solution r�pond � ce qui est demand� par SixSq mais elle reste toujours ouverte � plusieurs apports et extensions. Parmi ces �ventuelles �volutions, nous citons:

\begin{itemize}
\item \textbf{L'ajout des autres distributions de Hadoop tel que MapR et Cloudera} : avec cette �volution, notre solution offre � ses utilisateurs les trois distributions de Hadoop les plus utilis�es sur le march�.

\item \textbf{La s�paration entre le processus d'installation et celle de configuration} : avec cette �volution, nous pouvons cr�er des instances de VMs au niveau de Cloud avec tous les packages n�cessaires pour notre solution ce qui permet de minimiser le temps de d�ploiement.

\item \textbf{Avec la r�volution de Docker dans le monde du Cloud Computing, la 'Dockerisation' de notre solution devient une �tape d'importance primordiale} : avec cette �volution, le d�ploiement de notre solution devient beaucoup plus rapide qu'avant puisque le d�ploiement avec Docker se base sur des petits conteneurs l�gers et isol�s de machine ce qui implique que nous virtualisons des services, que nous combinons, et non une machine compl�te.

\end{itemize} 

%==============================================================================
\end{spacing}