\chapter{Conclusion G�n�rale et Perspectives}
%==============================================================================
\pagestyle{fancy}
\fancyhf{}
\fancyhead[R]{\bfseries\rightmark}
\fancyfoot[R]{\thepage}
\renewcommand{\headrulewidth}{0.5pt}
\renewcommand{\footrulewidth}{0pt}
\renewcommand{\chaptermark}[1]{\markboth{\MakeUppercase{\chaptername~\thechapter. #1 }}{}}
\renewcommand{\sectionmark}[1]{\markright{\thechapter.\thesection~ #1}}

\begin{spacing}{1.2}
%==============================================================================

C'est l'une des parties les plus importantes et pourtant les plus n�glig�es 
du rapport. Ce qu'on \underline{ne veut pas voir} ici, c'est combien ce stage vous a �t� b�n�fique, comment il vous a appris � vous int�grer, � conna�tre le monde du travail, etc.\\
Franchement, personne n'en a rien � faire, du moins dans cette partie. Pour cela, vous 
avez les remerciements et les d�dicaces, vous pourrez vous y exprimer � souhait.\\
La conclusion, c'est tr�s simple : c'est d'abord le r�sum� de ce que vous avez racont�
dans le rapport : vous reprenez votre contribution, en y ajoutant ici les outils que vous 
avez utilis�, votre mani�re de proc�der. Vous pouvez m�me mettre les difficult�s
rencontr�es. En deuxi�me lieu, on y met les perspectives du travail : ce qu'on pourrait 
ajouter � votre application, comment on pourrait l'am�liorer.

%==============================================================================
\end{spacing}